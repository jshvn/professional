%-------------------------------------------------------------------------------
% SECTION TITLE
%-------------------------------------------------------------------------------
% This manual line spacing is needed to prevent the header from being cut off
% If previous sections are updated, this will need to be modified or removed
\needspace{50\baselineskip}
\cvsection{Coursework}

%-------------------------------------------------------------------------------
% CONTENT
%-------------------------------------------------------------------------------
\begin{cvcourses}

%---------------------------------------------------------
%   \cvcourse
%     {UC Los Angeles} % Category
%     {Seminar on Current Topics (CS201), System Design/Architecture (CS259), Machine Learning (CS260), Databases (CS143), Health Analytics (CS239), Data Science (CS249), Computer Science Classics (CS259), Computer Security (CS259), Algorithmic Machine Learning (CS289ML), Software Engineering (CS130), Artificial Intelligence (CS161), Algorithms (CS180), Intellectual Property for Technology Entrepreneurs and Managers (EE293), Cryptography (CS282A)} % Skills

% %---------------------------------------------------------
%   \cvcourse
%     {UC Davis} % Category
%     {Semiconductor Device Physics I (EEC140A), Digital Systems I/II (EEC180A/B), Physics: Quantum, Nuclear, Optic (PHYS09D), Analog Electronic Circuits (EEC110A), Embedded Systems (EEC172), Operating Systems (ECS150), Computer Architecture (EEC170), NATCAR Design Project (EEC195A/B), Issues in Engineering Design (EEC196), Computer Networks \& Projects (EEC173A/B), Algorithm Design (ECS122A), Probability Theory (STA131A), Research Group/Directed Study (EEC190C/198), Network Architecture and Resource Management (EEC273)} % Skills

% %---------------------------------------------------------
%   \cvcourse
%     {UC Riverside} % Category
%     {Object Oriented Programming I/II (CS10/12), Discrete Structures I/II (CS11/111), Data Structures and Algorithms (CS14), Software Construction (CS100), Calculus I/II/III (MATH9A/B/C), Physics: Mechanics, Thermodynamics, Electrostatics, Fluid Dynamics (PHYS40A/B/C), Multivariable Calculus (MATH10A), Differential Equations (MATH46), Electrical Circuit Analysis I/II (EE01A/B), Sequential and Prepositional Logic (PHIL08), Linear Algebra/Matlab (MATH113), C++11 (CS290), Machine Organization/Assembly Language (CS61)} % Skills

%---------------------------------------------------------
  \cvcourselist
    {UC Los Angeles} % Category
    % Course descriptions sourced from: https://registrar.ucla.edu/academics/course-descriptions?search=Computer+Science
    {
      \begin{multicols}{3}
        \begin{itemize}
          \item Seminar on Current Topics (CS201) % Seminar, four hours; outside study, eight hours. Preparation: completion of major field examination in computer science. Current computer science research into theory of, analysis and synthesis of, and applications of information processing systems. Each member completes one tutorial and one or more original pieces of work in one specialized area. May be repeated for credit. Letter grading. Units: 4.0
          \item System Design/Architecture (CS259) % Lecture, four hours; outside study, eight hours. Review of current literature in area of computer science system design in which instructor has developed special proficiency as consequence of research interests. Students report on selected topics. May be repeated for credit with topic change. Letter grading. Units: 4.0
          \item Machine Learning (CS260) % Lecture, four hours; discussion, two hours; outside study, six hours. Recommended requisite: course 180. Problems of identifying patterns in data. Machine learning allows computers to learn potentially complex patterns from data and to make decisions based on these patterns. Introduction to fundamentals of this discipline to provide both conceptual grounding and practical experience with several learning algorithms. Techniques and examples used in areas such as healthcare, financial systems, commerce, and social networking. Letter grading. Units: 4.0
          \item Databases (CS143) % Lecture, four hours; outside study, eight hours. Requisite: course 143. Theoretical and technological foundation of Intelligent Database Systems, that merge database technology, knowledge-based systems, and advanced programming environments. Rule-based knowledge representation, spatio-temporal reasoning, and logic-based declarative querying/programming are salient features of this technology. Other topics include object-relational systems and data mining techniques. Letter grading. Units: 4.0
          \item Health Analytics (CS239) % Lecture, four hours; outside study, eight hours. Enforced requisites: courses 31, 180. Recommended: statistics and probability, numerical methods, knowledge in programming languages. Applied data analytics course, with focus on healthcare applications. How to properly generate and analyze health data. Project-based course to learn about best practices in health data collection and validation. Exploration of various machine learning and data analytic tools to learn underlying structure of datasets to solve healthcare problems. Different machine learning concepts and algorithms, statistical models, and building of data-driven models. Big data analytics and tools for handling structured, unstructured, and semistructured datasets. Letter grading. Units: 4.0
          \item Data Science (CS249) % Lecture, four hours; outside study, eight hours. Review of current literature in area of data structures in which instructor has developed special proficiency as consequence of research interests. Students report on selected topics. May be repeated for credit with consent of instructor. Letter grading. Units: 4.0
          \item Computer Science Classics (CS259) % Lecture, four hours; outside study, eight hours. Review of current literature in area of computer science system design in which instructor has developed special proficiency as consequence of research interests. Students report on selected topics. May be repeated for credit with topic change. Letter grading. Units: 4.0
          \item Computer Security (CS259) % Lecture, four hours; outside study, eight hours. Basic and research material on computer security. Topics include basic principles and goals of computer security, common security tools, use of cryptographic protocols for security, security tools (firewalls, virtual private networks, honeypots), virus and worm protection, security assurance and testing, design of secure programs, privacy, applying security principles to realistic problems, and new and emerging threats and security tools. Letter grading. Units: 4.0
          \item Algorithmic Machine Learning (CS289ML) % Lecture, four hours; outside study, eight hours. In-depth examination of handful of ubiquitous algorithms in machine learning. Covers several classical tools in machine learning but more emphasis on recent advances and developing efficient and provable algorithms for learning tasks. Topics include low-rank approximations, online learning, multiplicative weights framework, mathematical optimization, outlier-robust algorithms, streaming algorithms. S/U or letter grading. Units: 4.0
          \item Software Engineering (CS130) % Lecture, four hours; discussion, two hours. Recommended preparation for undergraduate students: prior software engineering course. Required preparation for graduate students: undergraduate-level knowledge of data structures and object-oriented program languages. As software systems become increasingly large and complex, automated software engineering analysis and development tools play important role in various software engineering tasks, such as design, construction, evolution, and testing and debugging of software systems. Introduction to foundations, techniques, tools, and applications of automated software engineering technology. Development, extension, and evaluation of mini automated software engineering analysis tool and assessment of how tool fits into software development process. Introduction to current research topics in automated software engineering. S/U or letter grading. Units: 4.0
          \item Cryptography (CS282A) % Lecture, four hours; discussion, two hours; outside study, six hours. Preparation: knowledge of basic probability theory. Enforced requisite: course 180. Introduction to cryptography, computer security, and basic concepts and techniques. Topics include notions of hardness, one-way functions, hard-core bits, pseudorandom generators, pseudorandom functions and pseudorandom permutations, semantic security, public-key and private-key encryption, key-agreement, homomorphic encryption, private information retrieval and voting protocols, message authentication, digital signatures, interactive proofs, zero-knowledge proofs, collision-resistant hash functions, commitment protocols, and two-party secure computation with static security. Letter grading. Units: 4.0
          \item Artificial Intelligence (CS161) % Lecture, four hours; laboratory, two hours; outside study, six hours. Enforced requisite: course 180. Introduction to fundamental problem solving and knowledge representation paradigms of artificial intelligence. Introduction to Lisp with regular programming assignments. State-space and problem reduction methods, brute-force and heuristic search, planning techniques, two-player games. Knowledge structures including predicate logic, production systems, semantic nets and primitives, frames, scripts. Special topics in natural language processing, expert systems, vision, and parallel architectures. Letter grading. Units: 4.0
          \item Algorithms (CS180) % Lecture, four hours; discussion, two hours; outside study, six hours. Enforced requisites: course 32, Mathematics 61. Designed for junior/senior Computer Science majors. Introduction to design and analysis of algorithms. Design techniques: divide-and-conquer, greedy method, dynamic programming; selection of prototypical algorithms; choice of data structures and representations; complexity measures: time, space, upper, lower bounds, asymptotic complexity; NP-completeness. Letter grading. Units: 4.0
          \item Intellectual Property for Technology Entrepreneurs and Managers (EE293) % Lecture, four hours; outside study, eight hours. Introduction to intellectual property law and its application to technology entrepreneurship and management. Topics include patents, copyrights, trademarks, trade secrets, and licensing. Emphasis on practical aspects of intellectual property protection and management in technology-driven industries. Letter grading. Units: 4.0
        \end{itemize}
      \end{multicols}
    }
%---------------------------------------------------------
  \cvcourselist
    {UC Davis} % Category
    {
      \begin{multicols}{3}
        \begin{itemize}
          \item Semiconductor Device Physics I (EEC140A) % Lecture, three hours; discussion, one hour. Introduction to semiconductor materials and devices. Topics include energy bands, charge carriers, carrier transport, p-n junctions, and basic device structures. Letter grading. Units: 4.0
          \item Digital Systems I/II (EEC180A/B) % Lecture, three hours; laboratory, three hours. Introduction to digital logic design and implementation. Topics include combinational and sequential logic, finite state machines, and digital system design. Letter grading. Units: 4.0 each
          \item Physics: Quantum, Nuclear, Optic (PHYS09D) % Lecture, three hours; discussion, one hour. Introduction to quantum mechanics, nuclear physics, and optics. Topics include wave-particle duality, atomic structure, and optical phenomena. Letter grading. Units: 4.0
          \item Analog Electronic Circuits (EEC110A) % Lecture, three hours; laboratory, three hours. Analysis and design of analog electronic circuits. Topics include operational amplifiers, diodes, and transistors. Letter grading. Units: 4.0
          \item Embedded Systems (EEC172) % Lecture, three hours; laboratory, three hours. Design and implementation of embedded systems. Topics include microcontrollers, real-time operating systems, and interfacing techniques. Letter grading. Units: 4.0
          \item Operating Systems (ECS150) % Lecture, three hours; discussion, one hour. Introduction to operating system concepts. Topics include process management, memory management, file systems, and concurrency. Letter grading. Units: 4.0
          \item Computer Architecture (EEC170) % Lecture, three hours; discussion, one hour. Introduction to computer architecture. Topics include instruction set design, pipelining, memory hierarchy, and parallelism. Letter grading. Units: 4.0
          \item NATCAR Design Project (EEC195A/B) % Laboratory, six hours. Design and implementation of autonomous vehicles. Topics include sensor integration, control algorithms, and real-time processing. Letter grading. Units: 3.0 each
          \item Issues in Engineering Design (EEC196) % Lecture, one hour. Discussion of ethical, social, and professional issues in engineering design. Topics include teamwork, communication, and project management. Letter grading. Units: 1.0
          \item Computer Networks \& Projects (EEC173A/B) % Lecture, three hours; laboratory, three hours. Introduction to computer networking concepts and protocols. Topics include TCP/IP, routing, and network security. Letter grading. Units: 4.0 each
          \item Algorithm Design (ECS122A) % Lecture, three hours; discussion, one hour. Introduction to algorithm design and analysis. Topics include divide-and-conquer, dynamic programming, and graph algorithms. Letter grading. Units: 4.0
          \item Probability Theory (STA131A) % Lecture, three hours; discussion, one hour. Introduction to probability theory. Topics include random variables, probability distributions, and statistical inference. Letter grading. Units: 4.0
          \item Research Group/Directed Study (EEC190C/198) % Independent study under faculty supervision. Topics vary based on research interests. Letter grading. Units: 1.0-5.0
          \item Network Architecture and Resource Management (EEC273) % Lecture, three hours; discussion, one hour. Advanced topics in network architecture and resource management. Topics include quality of service, traffic engineering, and network optimization. Letter grading. Units: 4.0
        \end{itemize}
      \end{multicols}
    }
%---------------------------------------------------------
  \cvcourselist
    {UC Riverside} % Category
    {
      \begin{multicols}{3}
        \begin{itemize}
          \item Object Oriented Programming I/II (CS10/12) % Lecture, three hours; laboratory, one hour. Introduction to object-oriented programming concepts and techniques. Topics include classes, objects, inheritance, polymorphism, and exception handling. Letter grading. Units: 4.0 each
          \item Discrete Structures I/II (CS11/111) % Lecture, three hours; discussion, one hour. Introduction to discrete mathematics for computer science. Topics include logic, set theory, combinatorics, graph theory, and algorithms. Letter grading. Units: 4.0 each
          \item Data Structures and Algorithms (CS14) % Lecture, three hours; discussion, one hour. Introduction to data structures and algorithm design. Topics include arrays, linked lists, trees, graphs, sorting, and searching algorithms. Letter grading. Units: 4.0
          \item Software Construction (CS100) % Lecture, three hours; laboratory, one hour. Introduction to software development practices and tools. Topics include version control, debugging, testing, and build systems. Letter grading. Units: 4.0
          \item Calculus I/II/III (MATH9A/B/C) % Lecture, three hours; discussion, one hour. Introduction to differential and integral calculus. Topics include limits, derivatives, integrals, and series. Letter grading. Units: 4.0 each
          \item Physics: Mechanics, Thermodynamics, Electrostatics, Fluid Dynamics (PHYS40A/B/C) % Lecture, three hours; discussion, one hour. Introduction to classical physics. Topics include mechanics, thermodynamics, electrostatics, and fluid dynamics. Letter grading. Units: 4.0 each
          \item Multivariable Calculus (MATH10A) % Lecture, three hours; discussion, one hour. Introduction to calculus of functions of several variables. Topics include partial derivatives, multiple integrals, and vector calculus. Letter grading. Units: 4.0
          \item Differential Equations (MATH46) % Lecture, three hours; discussion, one hour. Introduction to ordinary differential equations. Topics include first-order equations, linear equations, and applications. Letter grading. Units: 4.0
          \item Electrical Circuit Analysis I/II (EE01A/B) % Lecture, three hours; laboratory, one hour. Introduction to electrical circuit analysis. Topics include circuit laws, network theorems, and transient analysis. Letter grading. Units: 4.0 each
          \item Sequential and Prepositional Logic (PHIL08) % Lecture, three hours; discussion, one hour. Introduction to formal logic. Topics include propositional and predicate logic, logical equivalence, and proof techniques. Letter grading. Units: 4.0
          \item Linear Algebra/Matlab (MATH113) % Lecture, three hours; discussion, one hour. Introduction to linear algebra and its applications. Topics include matrices, vector spaces, eigenvalues, and numerical methods using Matlab. Letter grading. Units: 4.0
          \item C++11 (CS290) % Lecture, three hours; laboratory, one hour. Advanced topics in C++ programming. Topics include modern C++ features, templates, and standard library. Letter grading. Units: 4.0
          \item Machine Organization/Assembly Language (CS61) % Lecture, three hours; laboratory, one hour. Introduction to computer organization and assembly language programming. Topics include instruction sets, memory hierarchy, and I/O systems. Letter grading. Units: 4.0
        \end{itemize}
      \end{multicols}
    }
%---------------------------------------------------------
\end{cvcourses}
