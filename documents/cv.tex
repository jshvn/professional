%!TEX TS-program = xelatex
%!TEX encoding = UTF-8 Unicode
% Awesome CV LaTeX Template for CV/Resume
%
% This template has been downloaded from:
% https://github.com/posquit0/Awesome-CV
%
% Author:
% Claud D. Park <posquit0.bj@gmail.com>
% http://www.posquit0.com
%
% Template license:
% CC BY-SA 4.0 (https://creativecommons.org/licenses/by-sa/4.0/)
%


%-------------------------------------------------------------------------------
% CONFIGURATION
%-------------------------------------------------------------------------------
%-------------------------------------------------------------------------------
% CONFIGURATIONS
%-------------------------------------------------------------------------------
% A4 paper size by default, use 'letterpaper' for US letter
\documentclass[11pt, a4paper]{awesome-cv}

% Configure page margins with geometry
\geometry{left=1.4cm, top=.8cm, right=1.4cm, bottom=1.8cm, footskip=.5cm}

% Color for highlights
% Awesome Colors: awesome-emerald, awesome-skyblue, awesome-red, awesome-pink, awesome-orange
%                 awesome-nephritis, awesome-concrete, awesome-darknight
\colorlet{awesome}{awesome-darknight}
% Uncomment if you would like to specify your own color
% \definecolor{awesome}{HTML}{CA63A8}

% Colors for text
% Uncomment if you would like to specify your own color
% \definecolor{darktext}{HTML}{414141}
% \definecolor{text}{HTML}{333333}
% \definecolor{graytext}{HTML}{5D5D5D}
% \definecolor{lighttext}{HTML}{999999}
% \definecolor{sectiondivider}{HTML}{5D5D5D}

% Set false if you don't want to highlight section with awesome color
\setbool{acvSectionColorHighlight}{false}

% If you would like to change the social information separator from a pipe (|) to something else
\renewcommand{\acvHeaderSocialSep}{\quad\textbar\quad}

% PDF Metadata for curriculum vitae
\hypersetup{%
  pdftitle={Josh Vaughen's Curriculum Vitae},
  pdfauthor={Josh Vaughen, Machine Learning Engineer},
  pdfsubject={Curriculum vitae for Josh Vaughen},
  pdfkeywords={Curriculum vitae, Machine Learning Engineer, Josh Vaughen}
}

%-------------------------------------------------------------------------------
% HEADER FOR PERSONAL INFORMATION
%-------------------------------------------------------------------------------
%-------------------------------------------------------------------------------
%	PERSONAL INFORMATION
%	Comment any of the lines below if they are not required
%-------------------------------------------------------------------------------
% Available options: circle|rectangle,edge/noedge,left/right
% \photo[circle,noedge,right]{./resume/media/profile}
\name{Josh}{Vaughen}
% Including the location here in the \position is a temporary hack. TODO<jshvn>: update this to be included in \address
\position{Machine Learning Engineer{\enskip•\enskip}Seattle, WA}
% \address{Seattle, WA}

\mobile{+1 916 303 3057}
\email{josh@vaughen.net}
%\dateofbirth{January 1st, 1970}
\homepage{ijosh.com}
% \github{jshvn}
% \linkedin{ijosh}
% \gitlab{gitlab-id}
% \stackoverflow{SO-id}{SO-name}
% \twitter{@twit}
% \skype{skype-id}
% \reddit{reddit-id}
% \medium{madium-id}
% \kaggle{kaggle-id}
% \hackerrank{hackerrank-id}
% \googlescholar{googlescholar-id}{name-to-display}
%% \firstname and \lastname will be used
% \googlescholar{googlescholar-id}{}
%\extrainfo{Seattle, WA}

% \quote{``Be the change that you want to see in the world."}


%-------------------------------------------------------------------------------
\begin{document}

% Print the header with above personal information
% Give optional argument to change alignment(C: center, L: left, R: right)
\makecvheader[L]

% Print the footer with 3 arguments(<left>, <center>, <right>)
% Leave any of these blank if they are not needed
\makecvfooter
  {\today}
  {Josh Vaughen{\enskip•\enskip}Curriculum Vitae written in {\LaTeX{}}}
  {\thepage}


%-------------------------------------------------------------------------------
%	CV/RESUME CONTENT
%	Each section is imported separately, open each file in turn to modify content
%-------------------------------------------------------------------------------

%-------------------------------------------------------------------------------
%	SECTION TITLE
%-------------------------------------------------------------------------------
\cvsection{Overview}


%-------------------------------------------------------------------------------
%	CONTENT
%-------------------------------------------------------------------------------
\begin{cvparagraph}

%---------------------------------------------------------
Josh is a Machine Learning Engineer with experience developing and deploying models for a variety of applications spanning tens of millions of users. 
He has strong communication and collaboration skills with deep experience working cross-functionally with product, engineering, and business teams. 
He specializes in revenue optimization, user engagement and retention, and in-product and email personalization.

Josh is seeking an opportunity to expand his skills with particular focus on deploying and maintaining models at web scale.

\end{cvparagraph}

%-------------------------------------------------------------------------------
%	SECTION TITLE
%-------------------------------------------------------------------------------
\cvsection{Work Experience}


%-------------------------------------------------------------------------------
%	CONTENT
%-------------------------------------------------------------------------------
\begin{cventries}

%---------------------------------------------------------
\cventry
    {World Traveler} % Job title
    {Sabbatical} % Organization
    {Earth} % Location
    {Apr. 2023 - Present} % Date(s)
    {Took a sabbatical for personal growth and exploration, solo-traveling to 50+ countries across 6 continents following a life-altering medical event.}
    {}

%---------------------------------------------------------
\cventry
    {Machine Learning Engineer IV} % Job title
    {Adobe} % Organization
    {Seattle, Washington} % Location
    {Jan. 2020 - Apr. 2023} % Date(s)
    {Developed models and supporting data engineering systems to enhance user onboarding experiences, retention efforts, and upsell strategies.} % Overall description
    {
      \begin{cvitems} % Description(s) of tasks/responsibilities
        \item {Constructed latent variable model to improve new user onboarding, leading to increase in activation and subscription rates}
        \item {Architected a containerized data engineering production system, leveraging MLflow concepts, to deliver tens of millions of predictions monthly}
        \item {Utilized XGBoost to identify primary factors that influenced customer attrition, guiding retention strategies with product teams}
        \item {Refined an implementation of the apriori algorithm to handle larger datasets, optimizing for both computational complexity and rule granularity}
        \item {Implemented a shared Python library, Docker containers, and standard data model to unify and support team operational activities across projects}
      \end{cvitems}
    }

%---------------------------------------------------------
\cventry
    {Data Scientist III} % Job title
    {Adobe} % Organization
    {San Jose, California} % Location
    {Jan. 2018 - Dec. 2019} % Date(s)
    {Deployed and analyzed AB tests, developed pricing models, and led activation and retention modeling for Acrobat.} % Overall description
    {
      \begin{cvitems} % Description(s) of tasks/responsibilities
        \item {Developed a novel approach to upsell modeling, leveraging a combination of collaborative filtering and content-based filtering}
        \item {Used conjoint analysis to improve feature bundling and pricing, leading to increase in revenue and new SKU offerings}
        \item {Applied cohort and churn analysis techniques on users, features, and subscriptions to uncover activation and retention factors}
        \item {Designed, deployed, and analyzed AB tests focused on revenue impact, materially increasing subscription counts and ARR}
      \end{cvitems}
    }

%---------------------------------------------------------
\cventry
    {Data Scientist II} % Job title
    {Adobe} % Organization
    {San Jose, California} % Location
    {Aug. 2016 - Jan. 2018} % Date(s)
    {Implemented a KPI reporting and analytics framework for Acrobat, Sign, and Scan; including data engineering and an alert system for detected issues.} % Overall description
    {
      \begin{cvitems} % Description(s) of tasks/responsibilities
        \item {Conducted in-product AB tests designed to increase sign-in rates, resulting in the majority of all users transitioning from anonymous to signed-in}
        \item {Created an anomoly detection system leveraging ARIMA and STL decomposition methods, identified several previously unknown major issues}
        \item {Developed a unified KPI reporting framework and system based on SMART goals that was adopted across the organization to drive data strategy}
      \end{cvitems}
    }

%---------------------------------------------------------
\end{cventries}

%-------------------------------------------------------------------------------
%	SECTION TITLE
%-------------------------------------------------------------------------------
%\cvsection{Work Experience}

% For the purposes of interview preparation, we want to include on each bullet
% point in the resume a commented section with the following:
%
% - problem definition
% - possible approaches
% - what approach was taken and why
% - what technologies were used in that approach and why
% - results of that approach
% - what was learned from that experience
% - what would be done differently next time
% - what was the business impact of that work
% 
% This will be filled out for each of the below bullet points in the resume

%-------------------------------------------------------------------------------
%	CONTENT
%-------------------------------------------------------------------------------
\begin{cventries}

%---------------------------------------------------------
\cventry
    {Graduate Teaching Assistant} % Job title
    {University of California, Los Angeles} % Organization
    {Los Angeles, California} % Location
    {Mar. 2015 - Jun. 2016} % Date(s)
    {I was a Graduate Teaching Assistant for CS35L Software Construction Laboratory. I led lectures and labs several times a week, as well as held recurring office hours and review sessions before exams.} % Overall description
    {
      \begin{cvitems} % Description(s) of tasks/responsibilities
        \item {Led lecture 3x weekly, office hours 2x weekly}
        \item {Wrote and graded homework and exams}
      \end{cvitems}
    }

%---------------------------------------------------------
\cventry
    {Software Engineering Intern - Mobile App Development} % Job title
    {Adobe} % Organization
    {San Jose, California} % Location
    {Jun. 2015 - Sep. 2015} % Date(s)
    {Joined the Acrobat iOS team to grow my knowledge of iOS and watchOS. Developed a prototype Apple Watch integration with Acrobat that was ahead of its time. Worked on initial Dropbox integration with Acrobat.} % Overall description
    {
      \begin{cvitems} % Description(s) of tasks/responsibilities
        \item {Proposed and developed a prototype presentation control companion application for Acrobat on the Apple Watch}
        \item {Utilized Core Motion API to detect and react to hand gestures, allowing user to control their presentation without button pressing}
        \item {Worked on initial Dropbox cloud provider filesystem integration within Acrobat iOS}
        \item {Winner of the 2015 Intern Project Expo in San Jose for the second year in a row for Apple Watch work}
      \end{cvitems}
    }

%---------------------------------------------------------
\cventry
    {Software Engineering Intern - Mobile App Development} % Job title
    {Adobe} % Organization
    {San Jose, California} % Location
    {Jun. 2014 - Sep. 2014} % Date(s)
    {Joined the Adobe Sign team to develop the iOS application. Learned iOS and general software development methodologies on the fly.} % Overall description
    {
      \begin{cvitems} % Description(s) of tasks/responsibilities
        \item {Implemented four new features, fixed several existing bugs, improved usability of several interfaces}
        \item {Designed, implemented, and demoed a new prototype Sign UI/UX for my HackWeek project}
        \item {Wrote automated tests for iPhone/iPad on new Appium-based automation framework}
        \item {Completed Green Belt security training for C/C++ developers}
        \item {Winner of the 2014 Intern Project Expo in San Jose for UI/UX and automation work}
      \end{cvitems}
    }

%---------------------------------------------------------
\cventry
    {Software QA Engineering Intern} % Job title
    {Aruba Networks, an HPE Company} % Organization
    {Sunnyvale, California} % Location
    {Jun. 2013 - Dec. 2013} % Date(s)
    {Joined the hardware automation team at Aruba to develop an automated testing platform based on commodity hardware (Raspberry Pi, Roomba robots) and a pan-tilt servo system. This platform was used to automate physical world interaction and 802.11 RF tests.} % Overall description
    {
      \begin{cvitems} % Description(s) of tasks/responsibilities
        \item {Wrote a controlling wrapper API over existing Roomba microcode to simplify and more robustly control the platform}
        \item {Wrote orchestration platform and API to automate control of six free-roaming robots, effectively swarm control, including collision detection and avoidance}
        \item {Wrote controlling API for commodity pan-tilt servo system such that RF heatmaps could be profiled in multiple orientations via automation}
        \item {Successfully completed ACMA and ACMP certification courses}
        \item {Presented project outcomes at Aruba Tech Talk; was the Featured Intern on corporate blog}
      \end{cvitems}
    }

%---------------------------------------------------------
\cventry
    {Software Engineering Intern} % Job title
    {Galil Motion Control} % Organization
    {Rocklin, California} % Location
    {Jun. 2012 - Sep. 2012} % Date(s)
    {Joined the Research and Development Engineering team to build an automated firmware verification and testing system. I also developed tooling frameworks for the now-released GalilSuite application.} % Overall description
    {
      \begin{cvitems} % Description(s) of tasks/responsibilities
        \item {Rewrote existing test software reducing execution time in half}
        \item {Wrote automated firmware verification system with ability to simultaneously support tens of controllers}
        \item {Wrote portable (Windows/Linux) tool for GalilSuite server API audits}
      \end{cvitems}
    }

%---------------------------------------------------------
\end{cventries}

%-------------------------------------------------------------------------------
%	SECTION TITLE
%-------------------------------------------------------------------------------
\cvsection{Honors \& Awards}


%-------------------------------------------------------------------------------
%	SUBSECTION TITLE
%-------------------------------------------------------------------------------
\cvsubsection{UC Los Angeles Awards}


%-------------------------------------------------------------------------------
%	CONTENT
%-------------------------------------------------------------------------------
\begin{cvhonors}

%---------------------------------------------------------
  \cvhonor
    {2014 Great Lakes National Scholarship} % Award
    {Great Lakes Higher Education Corporation and Affiliates} % Organization
    {Los Angeles, CA} % Location
    {Jul. 2014} % Date(s)

%---------------------------------------------------------
\end{cvhonors}


%-------------------------------------------------------------------------------
%	SUBSECTION TITLE
%-------------------------------------------------------------------------------
\cvsubsection{UC Davis Awards}


%-------------------------------------------------------------------------------
%	CONTENT
%-------------------------------------------------------------------------------
\begin{cvhonors}

%---------------------------------------------------------
  \cvhonor
    {2014 Union Pacific Scholarship} % Award
    {University of California, Davis} % Organization
    {Davis, CA} % Location
    {May. 2014} % Date(s)

%---------------------------------------------------------
  \cvhonor
    {Elizabeth Power Wood Scholarship} % Award
    {University of California, Davis} % Organization
    {Davis, CA} % Location
    {Oct. 2012} % Date(s)

%---------------------------------------------------------
  \cvhonor
    {UC Davis Undergraduate Scholarship} % Award
    {University of California, Davis} % Organization
    {Davis, CA} % Location
    {Sep. 2012} % Date(s)

%---------------------------------------------------------
\end{cvhonors}


%-------------------------------------------------------------------------------
%	SUBSECTION TITLE
%-------------------------------------------------------------------------------
\cvsubsection{UC Riverside Awards}


%-------------------------------------------------------------------------------
%	CONTENT
%-------------------------------------------------------------------------------
\begin{cvhonors}

%---------------------------------------------------------
  \cvhonor
    {University Grant} % Award
    {University of California, Riverside} % Organization
    {Riverside, CA} % Location
    {Sep. 2011} % Date(s)

%---------------------------------------------------------
  \cvhonor
    {State Supplemental Grant} % Award
    {University of California, Riverside} % Organization
    {Riverside, CA} % Location
    {Sep. 2010} % Date(s)

%---------------------------------------------------------
\end{cvhonors}

%-------------------------------------------------------------------------------
%	SECTION TITLE
%-------------------------------------------------------------------------------
\cvsection{Certificates}


%-------------------------------------------------------------------------------
%	CONTENT
%-------------------------------------------------------------------------------
\begin{cvhonors}

%---------------------------------------------------------
  \cvhonor
    {Aruba Certified Mobility Associate (ACMA)} % Name
    {Aruba Networks} % Issuer
    % Need to dig up this credential ID #
    {} % Credential ID
    {July 2013} % Date(s)

%---------------------------------------------------------
  \cvhonor
    {Aruba Certified Mobility Professional (ACMP)} % Name
    {Aruba Networks} % Issuer
    % Need to dig up this credential ID #
    {} % Credential ID
    {July, 2013} % Date(s)

%---------------------------------------------------------
\end{cvhonors}

%%-------------------------------------------------------------------------------
%	SECTION TITLE
%-------------------------------------------------------------------------------
\cvsection{Presentation}


%-------------------------------------------------------------------------------
%	CONTENT
%-------------------------------------------------------------------------------
\begin{cventries}

%---------------------------------------------------------
  \cventry
    {Role} % Role
    {Event} % Event
    {Location} % Location
    {Dates} % Date(s)
    {} % Overall Description
    {
      \begin{cvitems} % Description(s)
        \item {Line item}
        \item {Line item}
      \end{cvitems}
    }

%---------------------------------------------------------
  \cventry
    {Role} % Role
    {Event} % Event
    {Location} % Location
    {Dates} % Date(s)
    {} % Overall Description
    {
    \begin{cvitems} % Description(s)
        \item {Line item}
        \item {Line item}
    \end{cvitems}
    }

%---------------------------------------------------------
\end{cventries}

%%-------------------------------------------------------------------------------
%	SECTION TITLE
%-------------------------------------------------------------------------------
\cvsection{Writing}


%-------------------------------------------------------------------------------
%	CONTENT
%-------------------------------------------------------------------------------
\begin{cventries}

%---------------------------------------------------------
  \cventry
    {Role} % Role
    {Title} % Title
    {Location} % Location
    {Dates} % Date(s)
    {
      \begin{cvitems} % Description(s)
        \item {Line item}
        \item {Line item}
      \end{cvitems}
    }

%---------------------------------------------------------
  \cventry
    {Role} % Role
    {Title} % Title
    {Location} % Location
    {Dates} % Date(s)
    {
        \begin{cvitems} % Description(s)
        \item {Line item}
        \item {Line item}
        \end{cvitems}
    }


%---------------------------------------------------------
\end{cventries}

%-------------------------------------------------------------------------------
%	SECTION TITLE
%-------------------------------------------------------------------------------
\cvsection{Professional Organizations}


%-------------------------------------------------------------------------------
%	CONTENT
%-------------------------------------------------------------------------------
\begin{cventries}

%---------------------------------------------------------
\cventry
    {Member, Secretary, Webmaster} % Position title
    {Tau Beta Pi} % Organization
    {Davis, CA} % Location
    {Jun. 2012 - Present} % Date(s)
    {Served as webmaster, rebuilding the Davis Tau Beta Pi website, and as secretary, coordinating communication between national headquarters and the local chapter. As an alumnus, continue to participate in volunteer events and chapter activities.} % Overall description
    {}

%---------------------------------------------------------
\cventry
    {Member} % Position title
    {Institute of Electrical and Electronics Engineers (IEEE)} % Organization
    {Riverside, CA} % Location
    {Sep. 2010 - Present} % Date(s)
    {Volunteered for IEEE Boy Scouts Merit Badge Day, Micromouse robotics workshops, peer tutoring, and various chapter activities.} % Overall description
    {}

%---------------------------------------------------------
\cventry
    {Member} % Position title
    {Association for Computing Machinery} % Organization
    {Riverside, CA} % Location
    {Sep. 2010 - Present} % Date(s)
    {Habitual peer tutoring, chapter professional events, and other chapter activities.} % Overall description
    {}

%---------------------------------------------------------
\cventry
    {Member} % Position title
    {Golden Key International} % Organization
    {Riverside, CA} % Location
    {Dec. 2011 - Present} % Date(s)
    {Member of Golden Key International Honor Society, recognizing academic excellence, leadership, and commitment to service.} % Overall description
    {}

%---------------------------------------------------------
\cventry
    {Member} % Position title
    {National Society of Collegiate Scholars} % Organization
    {Riverside, CA} % Location
    {Feb. 2011 - Present} % Date(s)
    {Member of the National Society of Collegiate Scholars, recognizing academic excellence and leadership among high-achieving students.} % Overall description
    {}

%---------------------------------------------------------
\end{cventries}
%-------------------------------------------------------------------------------
%	SECTION TITLE
%-------------------------------------------------------------------------------
\cvsection{Extracurricular Activities}


%-------------------------------------------------------------------------------
%	CONTENT
%-------------------------------------------------------------------------------
\begin{cventries}

%---------------------------------------------------------
  \cventry
    {Volunteer Interviewer} % Affiliation/role
    {Year Up} % Organization/group
    {Seattle, WA} % Location
    {Nov. 2021} % Date(s)
    {Volunteered with Year Up to conduct mock interviews, helping underrepresented youth gain practical experience and prepare for successful careers.} % Overall Description
    {}

%---------------------------------------------------------
  \cventry
    {Technical Volunteer} % Affiliation/role
    {Adobe Tech Summit} % Organization/group
    {San Jose, CA} % Location
    {Feb 2017. - Feb 2019} % Date(s)
    {Volunteered to run the Adobe Tech Summit Robot Challenge, providing technical guidance and support to contestants on rules and development.} % Overall Description
    {}

%---------------------------------------------------------
  \cventry
    {Team Mentor} % Affiliation/role
    {First Robotics} % Organization/group
    {Lincoln, CA} % Location
    {Jan 2009. - Apr 2014} % Date(s)
    {Founded and mentored FIRST Robotics team 3257, guiding high school students in designing and programming robots for the FRC competitions.} % Overall Description
    {
        \begin{cvitems} % Description(s) of tasks/responsibilities
            \item {Volunteered hundreds of hours to instruct high school students in robotics concepts as part of the FIRST Robotics Competition (FRC) program}
            \item {Promoted professionalism and soft skills among students throughout the six-week build season, fostering effective teamwork and leadership}
            \item {Supported the Sacramento Regional Competition by facilitating pit organization and ensuring smooth event operations for the FRC community}
        \end{cvitems}
    }

%---------------------------------------------------------
  \cventry
    {Volunteer Instructor} % Affiliation/role
    {IEEE} % Organization/group
    {Riverside, CA} % Location
    {Apr. 2012} % Date(s)
    {Volunteer for IEEE Merit Badge Day, instructing Boy Scouts in areas of engineering, energy, robotics, electronics, and more for their merit badges.} % Overall Description
    {}

%---------------------------------------------------------
\end{cventries}

%-------------------------------------------------------------------------------
%	SECTION TITLE
%-------------------------------------------------------------------------------
\cvsection{Skills}


%-------------------------------------------------------------------------------
%	CONTENT
%-------------------------------------------------------------------------------
\begin{cvskills}

%---------------------------------------------------------
  \cvskill
    {Tools} % Category
    {MLflow, Tensorflow, Scikit-learn, Tableau, VSCode, AWS (s3, ec2), DO, git, Docker, k8s, Hive, Spark} % Skills

%---------------------------------------------------------
  \cvskill
    {Languages} % Category
    {Python, R, SQL, C++, Bash/ZSH, {\LaTeX{}}} % Skills

%---------------------------------------------------------
  \cvskill
    {Algorithms} % Category
    {XGBoost, KNN, SVM, linear \& logistic regression, k-means, apriori, recommender systems, LSTM, RNN} % Skills

%---------------------------------------------------------
  \cvskill
    {Techniques} % Category
    {Conjoint analysis, ARIMA, STL decomposition, PCA, A/B testing, ETL, CI/CD} % Skills

%---------------------------------------------------------
  \cvskill
    {Soft Skills} % Category
    {Cross-functional team leadership, public speaking, technical writing, mentorship and coaching, hiring} % Skills

%---------------------------------------------------------
\end{cvskills}

%-------------------------------------------------------------------------------
%	SECTION TITLE
%-------------------------------------------------------------------------------
% We want this commented out because otherwise two "Education" headers 
% would appear on the CV
%\cvsection{Education}


%-------------------------------------------------------------------------------
%	CONTENT
%-------------------------------------------------------------------------------
\begin{cventries}

%---------------------------------------------------------
  \cventry
    {B.S. in Computer Engineering} % Degree
    {University of California, Riverside} % Institution
    {Riverside, California} % Location
    {Sep. 2010 - Jun. 2012} % Date(s)
    {} % Overall Description
    {
      \begin{cvitems} % Description(s) bullet points
        \item {Dean's Honors List, ECE Department Student Advisor}
        \item {Tau Beta Pi member, IEEE member and volunteer, peer tutor}
      \end{cvitems}
    }
%---------------------------------------------------------
\end{cventries}

%-------------------------------------------------------------------------------
% SECTION TITLE
%-------------------------------------------------------------------------------
% This manual line spacing is needed to prevent the header from being cut off
% If previous sections are updated, this will need to be modified or removed
\needspace{50\baselineskip}
\cvsection{Coursework}

%-------------------------------------------------------------------------------
% CONTENT
%-------------------------------------------------------------------------------
\begin{cvcourses}

%---------------------------------------------------------
%   \cvcourse
%     {UC Los Angeles} % Category
%     {Seminar on Current Topics (CS201), System Design/Architecture (CS259), Machine Learning (CS260), Databases (CS143), Health Analytics (CS239), Data Science (CS249), Computer Science Classics (CS259), Computer Security (CS259), Algorithmic Machine Learning (CS289ML), Software Engineering (CS130), Artificial Intelligence (CS161), Algorithms (CS180), Intellectual Property for Technology Entrepreneurs and Managers (EE293), Cryptography (CS282A)} % Skills

% %---------------------------------------------------------
%   \cvcourse
%     {UC Davis} % Category
%     {Semiconductor Device Physics I (EEC140A), Digital Systems I/II (EEC180A/B), Physics: Quantum, Nuclear, Optic (PHYS09D), Analog Electronic Circuits (EEC110A), Embedded Systems (EEC172), Operating Systems (ECS150), Computer Architecture (EEC170), NATCAR Design Project (EEC195A/B), Issues in Engineering Design (EEC196), Computer Networks \& Projects (EEC173A/B), Algorithm Design (ECS122A), Probability Theory (STA131A), Research Group/Directed Study (EEC190C/198), Network Architecture and Resource Management (EEC273)} % Skills

% %---------------------------------------------------------
%   \cvcourse
%     {UC Riverside} % Category
%     {Object Oriented Programming I/II (CS10/12), Discrete Structures I/II (CS11/111), Data Structures and Algorithms (CS14), Software Construction (CS100), Calculus I/II/III (MATH9A/B/C), Physics: Mechanics, Thermodynamics, Electrostatics, Fluid Dynamics (PHYS40A/B/C), Multivariable Calculus (MATH10A), Differential Equations (MATH46), Electrical Circuit Analysis I/II (EE01A/B), Sequential and Prepositional Logic (PHIL08), Linear Algebra/Matlab (MATH113), C++11 (CS290), Machine Organization/Assembly Language (CS61)} % Skills

%---------------------------------------------------------
  \cvcourselist
    {UC Los Angeles} % Category
    {
      \begin{multicols}{3}
        \begin{itemize}
          \item Seminar on Current Topics (CS201)
          \item System Design/Architecture (CS259)
          \item Machine Learning (CS260)
          \item Databases (CS143)
          \item Health Analytics (CS239)
          \item Data Science (CS249)
          \item Computer Science Classics (CS259)
          \item Computer Security (CS259)
          \item Algorithmic Machine Learning (CS289ML)
          \item Software Engineering (CS130)
          \item Cryptography (CS282A)
          \item Artificial Intelligence (CS161)
          \item Algorithms (CS180)
          \item Intellectual Property for Technology Entrepreneurs and Managers (EE293)
        \end{itemize}
      \end{multicols}
    }
%---------------------------------------------------------
  \cvcourselist
    {UC Davis} % Category
    {
      \begin{multicols}{3}
        \begin{itemize}
          \item Semiconductor Device Physics I (EEC140A)
          \item Digital Systems I/II (EEC180A/B)
          \item Physics: Quantum, Nuclear, Optic (PHYS09D)
          \item Analog Electronic Circuits (EEC110A)
          \item Operating Systems (ECS150)
          \item Embedded Systems (EEC172)
          \item Computer Architecture (EEC170)
          \item NATCAR Design Project (EEC195A/B)
          \item Issues in Engineering Design (EEC196)
          \item Algorithm Design (ECS122A)
          \item Computer Networks \& Projects (EEC173A/B)
          \item Probability Theory (STA131A)
          \item Research Group/Directed Study (EEC190C/198)
          \item Network Architecture and Resource Management (EEC273)
        \end{itemize}
      \end{multicols}
    }
%---------------------------------------------------------
  \cvcourselist
    {UC Riverside} % Category
    {
      \begin{multicols}{3}
        \begin{itemize}
          \item Object Oriented Programming I/II (CS10/12)
          \item Discrete Structures I/II (CS11/111)
          \item Data Structures and Algorithms (CS14)
          \item Software Construction (CS100)
          \item Calculus I/II/III (MATH9A/B/C)
          \item Physics: Mechanics, Thermodynamics, Electrostatics, Fluid Dynamics (PHYS40A/B/C)
          \item Multivariable Calculus (MATH10A)
          \item Differential Equations (MATH46)
          \item Electrical Circuit Analysis I/II (EE01A/B)
          \item Sequential and Prepositional Logic (PHIL08)
          \item Linear Algebra/Matlab (MATH113)
          \item C++11 (CS290)
          \item Machine Organization/Assembly Language (CS61)
        \end{itemize}
      \end{multicols}
    }
%---------------------------------------------------------
\end{cvcourses}


%-------------------------------------------------------------------------------
\end{document}
